\documentclass[a4paper]{article}
\usepackage{float}
\usepackage[ruled,vlined,linesnumbered,algo2e]{algorithm2e}
\usepackage{amsmath,amssymb}
\usepackage{makecell}
\usepackage{tikz}
\usepackage[margin=0.8in]{geometry}
%\usepackage{biblatex} %Imports biblatex package
%\addbibresource{NLA.bib}
\DeclareMathOperator*{\argmax}{arg\,max}
\DeclareMathOperator*{\argmin}{arg\,min}
\DeclareMathOperator*{\diag}{diag}
\DeclareMathOperator*{\trace}{trace}
\DeclareMathOperator*{\sign}{sign}

\newcommand*\circled[1]{\tikz[baseline=(char.base)]{
		\node[shape=circle,draw=red,inner sep=1pt] (char) {#1};}}
\newcommand{\mbf}[1]{\mathbf{#1}}
\setlength\parindent{0pt} %% Do not touch this
\usepackage{amsthm}
\newtheorem{lemma}{Lemma}
\newtheorem{theorem}{Theorem}
%% -----------------------------
%% TITLE
%% -----------------------------
\title{Homework: Regularization} %% Assignment Title
%% Change "\today" by another date manually
%% -----------------------------
%% -----------------------------

%% %%%%%%%%%%%%%%%%%%%%%%%%%
\newcommand\myemptypage{
	\null
	\thispagestyle{empty}
	\addtocounter{page}{-1}
	\newpage
}

\usepackage{fancyhdr}
\usepackage{listings}
%\usepackage{siunitx}
\usepackage{hyperref}
\usepackage{etoolbox,refcount}
\usepackage{multicol}
\usepackage{tabularx,colortbl}
\usepackage{lastpage}
\usepackage{pgfplots}
\pgfplotsset{compat = newest}
\usepackage{biblatex} %Imports biblatex package
%\addbibresource{NLA.bib}
\usepackage{diagbox}
\pagestyle{fancy}
\usepackage{caption}
\usepackage{subcaption}
\usepackage{mathtools}
\usepackage{cleveref}
\usepackage[bottom]{footmisc}
\usepackage{comment}

\fancyhead[C]{}
\fancyhead[r]{}
%\renewcommand{\footrulewidth}{1pt}
\fancyfoot[L]{}
\fancyfoot[C]{}
\fancyfoot[R]{\thepage\ / \pageref*{LastPage}}

\begin{document}
	
	\begin{titlepage}
		
		\newcommand{\HRule}{\rule{\linewidth}{0.5mm}}  
		
		\begin{center}
			
			%\includegraphics[scale=0.3]{Documents/KUL_ENG.png}\\[1cm]
			\textsc{\LARGE Faculty of Engineering Science}\\[1.5cm] 
			
			
			\HRule \\[0.1cm]
			
			{\huge{ \bfseries Denoising and inpainting with wavelets}} 
			\vspace{5pt}
			{\\ \Large{\bfseries Wavelets with application in signal and Image processing}}
			\HRule \\[1cm]
			
			%\includegraphics[width=0.3\textwidth]{images1/INEOS.png}\\
			\vspace*{0.1cm}
			
			\begin{minipage}{0.4\textwidth}
				\textit{Auteur}
				\begin{flushleft} \large
					\begin{enumerate}
						\item[] \textsc{Thakur} Arnav\\
                        \item[] \textsc{Mohamed Yassin} Arman\\

						
					\end{enumerate} 
				\end{flushleft}
			\end{minipage}\\[1cm]
			\begin{minipage}{0.4\textwidth}
				\textit{Professors}
				\begin{flushleft}
					\begin{enumerate}
						\item[] Prof. Daan Huybrechs\\
					\end{enumerate} 
				\end{flushleft}
			\end{minipage}\\
			
			\vspace*{1.5cm}
			
			\includegraphics[width=0.5\textwidth]{Images/KUL_Eng_logo.png}\\
			{\Large{Academic year 2023 - 2024}}
			
		\end{center}
	\end{titlepage}
	
	\newpage
	
	\tableofcontents
	
	\newpage

    \section{Wavelet-based denoising}
 
	\subsection{A univariate functions with noise} \label{subsec:UniVariate}

    \subsubsection{Question 2.1}

	Consider the function:
    \begin{equation*}
	f(x) = (2+\cos{(x)}) |x| \sign{(x-1)}
	\end{equation*}
    Function sampled in a set of $N=1000$ points in the interval $[-2,2]$
    \begin{align*}
    	f_j = f(x_j) && x_j = -2 + 4\frac{j-1}{N-1}, \quad j=1,\ldots,N
    \end{align*}
    
    We compute the wavelet transform of the vector $f = \{f_j\}_{j=1}^N$ using the Daubechies 2 wavelet of 4 levels deep. boundary conditions used is: Symmetrization (half-point). These same parameters are used for the rest of \cref{subsec:UniVariate}.
    \begin{figure}[H]
	\centering
	\includegraphics[trim={3.5cm 8cm 4cm 9cm},clip,width=0.7\textwidth]{Images/Coefficents.pdf}
	\caption{Coefficients of Wavelet transform (4 levels deep, using the Daubechies 2, $N=1000$ sample points, between $[-2,2]$. DWT Extension Mode: Symmetrization (half-point)}
	\label{fig:Coeff}
    \end{figure}

    We can see in \footnote{To recreate \cref{fig:Coeff,fig:Delta=0,fig:Delta=0.1,fig:HardSoft,fig:OptiDelta} run \texttt{ex2point1.m}} \cref{fig:Coeff}, that the size of the coefficients decreases as $i$ increases. The meaning of this is that the coefficients of small size are of less importance to the reconstruction of the signal, and thus removing them will disproportionally remove noise.

    \subsubsection{Task 2.2}

	Noise is added to the signal $\tilde{f_i} = f_i + \epsilon \mathcal{U}(0,1)$ ($\hat{f}$ is the reconstructed $\tilde{f}$). We chose $\epsilon = \texttt{1e-1}$. \\
	Both hard thresholding:
	\begin{equation*}
			t_\delta = \begin{cases}
				0, \quad |x| < \delta\\
				x, \quad |x| \ge 0
		\end{cases}
	\end{equation*}
	And soft thresholding
	\begin{equation*}
			t_\delta = \begin{cases}
			0, \quad |x| < \delta\\
			\sign{(x)} (|x|-\delta), \quad |x| \ge 0
	\end{cases}
	\end{equation*}
	are going to be tested. $x$ are the wavelet coefficients.\\
	First let us choose $\delta = 0$, the results are in \cref{fig:Delta=0}. We can see in \cref{sub:ErrCoeffD=0}, that the highest error is in the first few coefficients. Which makes sense since we are in some sense moving the mean of the function $f$ by adding a value sampled from the uniform distribution, $\rightarrow$ lower frequencies should be more effected. The total error $E = \|f-\hat{f}\|_2 = 1.805$.\\
	
	Now let us set $\delta = \texttt{1e-1}$, the resulting figures are \cref{fig:Delta=0.1}. 3.35\% of the wavelet coefficients were set to 0, for both types of thresholding. \\
	For hard thresholding, $E = 1.5643$, which is indeed lower then the previous one. This seems manly due to the fact that $\hat{f}$ is smoother then $\tilde{f}$, see \cref{sub:ErrorD=0.1}. \\
	If knowledge of the mean of the noise is available, we can come put a better reconstruction of $f$ by $\hat{f}-mean(noise)$, in fact this is what will be done for the rest of \cref{subsec:UniVariate}, the resulting $E = 0.1982$, see \cref{sub:Delta=0.1Better}. In \cref{sub:Delta=0.1BetterSoft}, we can see the error when using soft thresholding, $E = 0.61$.

\begin{figure}[H]
	\centering
	\begin{subfigure}{0.49\textwidth}
	\centering
	%trim={<left> <lower> <right> <upper>}
	%\includegraphics[trim={4cm 8cm 4cm 8cm},clip,width=.49\textwidth]{Images/LQG_weighted4.pdf}
	\includegraphics[trim={4cm 8cm 4cm 8cm},clip,width=1\textwidth]{Images/FuncNoise.pdf}
	\caption{Plot of function, and it's noisy counterpart}
	\label{sub:FuncNoiseD=0}
\end{subfigure}
	\begin{subfigure}{0.49\textwidth}
		\centering
		%trim={<left> <lower> <right> <upper>}
		%\includegraphics[trim={4cm 8cm 4cm 8cm},clip,width=.49\textwidth]{Images/LQG_weighted4.pdf}
		\includegraphics[trim={4cm 8cm 4cm 8cm},clip,width=1\textwidth]{Images/CoeffDelta=0.pdf}
		\caption{The error between the real coefficients and noisy ones}
		\label{sub:ErrCoeffD=0}
	\end{subfigure}
	\begin{subfigure}{0.49\textwidth}
		\centering
		\includegraphics[trim={4cm 8cm 4cm 8cm},clip,width=1\textwidth]{Images/Delta=0.pdf}
		\caption{The error between real signal and noisy one}
		\label{sub:ErrorD=0}
	\end{subfigure}
	\caption{Plots for $\delta = 0$, black line is mean of noise. For hard thresholding}
	\label{fig:Delta=0}
\end{figure}

\begin{figure}[H]
	\centering
	\begin{subfigure}{0.49\textwidth}
		\centering
		%trim={<left> <lower> <right> <upper>}
		%\includegraphics[trim={4cm 8cm 4cm 8cm},clip,width=.49\textwidth]{Images/LQG_weighted4.pdf}
		\includegraphics[trim={4cm 8cm 4cm 8cm},clip,width=1\textwidth]{Images/CoeffDelta=0.1.pdf}
		\caption{The error between the real coefficients and reconstructed ones}
		\label{sub:ErrCoeffD=0.1}
	\end{subfigure}
	\begin{subfigure}{0.49\textwidth}
		\centering
		\includegraphics[trim={3.5cm 8cm 4cm 8cm},clip,width=1\textwidth]{Images/Delta=0.1.pdf}
		\caption{The error between real signal and reconstructed one}
		\label{sub:ErrorD=0.1}
	\end{subfigure}
	\caption{Plots for $\delta = 0.1$, black line is mean of noise. For hard thresholding}
	\label{fig:Delta=0.1}
\end{figure}

    \begin{figure}[H]
	\begin{subfigure}{0.49\textwidth}
	\centering
	\centering
\includegraphics[trim={3.5cm 8cm 4cm 9cm},clip,width=1\textwidth]{Images/Delta=0.1Better.pdf}
\caption{The error between real signal and reconstructed one, black line is mean of noise. For hard thresholding}
\label{sub:Delta=0.1Better}
\end{subfigure}
\begin{subfigure}{0.49\textwidth}
	\centering
\includegraphics[trim={3.5cm 8cm 4cm 9cm},clip,width=1\textwidth]{Images/Delta=0.1Soft.pdf}
\caption{The error between real signal and reconstructed one, black line is mean of noise. For soft thresholding}
\label{sub:Delta=0.1BetterSoft}
\end{subfigure}
\caption{}
\label{fig:HardSoft}
\end{figure}


    \subsubsection{Question 2.3}

	Now let us try to find the best parameter $\delta$ in order to minimize the noise. This will be done by simply checking MSE between the real and filtered coefficients $\sum_{i} (x_i - \hat{x}_i)^2$, for 101 different equidistant $\delta$'s in $[\texttt{1e-10},\texttt{1e0}]$. The results are shown in \cref{fig:OptiDelta}, the best delta according to MSE is 
	\begin{itemize}
		\item Hard threshold: $\delta = 0.501$ with an MSE=0.0026
		\item Soft threshold: $\delta = 0.0398$ with an MSE=0.0028
	\end{itemize}
	The reconstructed functions do have visibly less error then the noisy one \cref{sub:FuncNoiseD=0}.\\
	For hard thresholding, the error is especially noticeable for $x = 0$ (\cref{sub:BestDelta,sub:BestDeltaFunc}), probably due to the fact that $f$ is not differentiable due to $|x|$, and because of the very discontinuous nature of the thresholding this results in more error. For $x = 1$, we can also see a slight bump in error due the discontinuation of the $\sign$ function.\\
	For soft thresholding it would seem that the error is higher over the whole domain but lower in $x=0$, due to the fact that it is less discontinuous then hard thresholding. (\cref{sub:BestDeltaSoft,sub:BestDeltaSoftFunc}) \\
	
	It would seem that hard thresholding is overall smoother then soft except for $x=0$. 

    \begin{figure}[H]
	\begin{subfigure}{0.49\textwidth}
	\centering
\includegraphics[trim={3.5cm 8cm 4cm 9cm},clip,width=1\textwidth]{Images/DeltaOpti.pdf}
\caption{The error between real signal and reconstructed one, black line is mean of noise. $\delta = 0.501$. For Hard thresholding}
\label{sub:BestDelta}
\end{subfigure}
\begin{subfigure}{0.49\textwidth}
	\centering
\includegraphics[trim={3.5cm 8cm 4cm 9cm},clip,width=1\textwidth]{Images/DeltaOptiSoft.pdf}
\caption{The error between real signal and reconstructed one, black line is mean of noise.$\delta = 0.0398$. For soft threasholding}
\label{sub:BestDeltaSoft}
\end{subfigure}
	\begin{subfigure}{0.49\textwidth}
	\centering
	\includegraphics[trim={3.5cm 8cm 4cm 9cm},clip,width=1\textwidth]{Images/HardFunc.pdf}
	\caption{Plot of clean and reconstructed signal. $\delta = 0.501$. For Hard thresholding}
	\label{sub:BestDeltaFunc}
\end{subfigure}
\begin{subfigure}{0.49\textwidth}
	\centering
	\includegraphics[trim={3.5cm 8cm 4cm 9cm},clip,width=1\textwidth]{Images/SoftFunc.pdf}
	\caption{Plot of clean and reconstructed signal. $\delta = 0.0398$. For soft threasholding}
	\label{sub:BestDeltaSoftFunc}
\end{subfigure}
\caption{Plots for the recontructed functions, via different type of thresholding}
\label{fig:OptiDelta}
\end{figure}

    \subsection{Images with noise}

    \subsubsection{Task 2.4}
	An image is composed of 3 different matrices, each representing the value of the color red,blue and green. To denoise an image, the following steps will be executed:
	\begin{enumerate}
		\item Separate image (tensor) into 3 matrices representing the 3 RGB colors
		\item Compute the 2D wavelet transform for each of the 3 matrices
		\item Apply a thresholding scheme to the wavelet coefficients, for the 3 different colors
		\item Compute the inverse 2D wavelet transform, for the 3 different colors
		\item Recombine the denoised color matrices into 1 image (tensor)
	\end{enumerate}
	Since we are dealing with image denoising, the wavelet families to test on are Daubechies or biorthogonal CDF's. \\

	We will test the wavelet-based denoising scheme on a picture (\cref{sub:Bib}) for different thresholds (using hard thesholding), BiorSplines4.4 as wavelet and decomposition level of 4. The resulting figures are in \cref{fig:Bib}, \cref{tab:bib} summarizes the results. We can clearly see as the threshold is increased, the error (Frobenius) and compression ration increase as well. This is expected as a higher threshold results in more of the (signal) picture being thrown away.

\begin{table}[H]
	\centering
	\begin{tabular}{|l|l|l|l|}
	\hline
	$\delta$	& Compression ratio & $\|A-\tilde{A}\|_F$ & $\frac{\|A-\hat{A}\|_F}{\|A\|_F}$ \\ \hline
	4.32	& 1.95 & \texttt{2.37e+03} & \texttt{0.0097} \\ \hline
	43.205	& 9.73 & \texttt{2.38e+04} & \texttt{0.097} \\ \hline
	432.048	& 219.84 & \texttt{6.805e+04} & \texttt{0.28} \\ \hline
	\end{tabular}
	\caption{Table summarizing the results for different threshold values, tested on \cref{sub:Bib}. Using BiorSplines4.4 as wavelet and decomposition level of 4. $\hat{A}$ is the reconstructed image}
	\label{tab:bib}
\end{table}

\begin{figure}[H]
	\centering
	\begin{subfigure}{0.49\textwidth}
		\centering
		%trim={<left> <lower> <right> <upper>}
		%\includegraphics[trim={4cm 8cm 4cm 8cm},clip,width=.49\textwidth]{Images/LQG_weighted4.pdf}
		\includegraphics[trim={4cm 8cm 4cm 8cm},clip,width=1\textwidth]{Images/Bib.pdf}
		\caption{Picture of a library, original}
		\label{sub:Bib}
	\end{subfigure}
	\begin{subfigure}{0.49\textwidth}
	\centering
	%trim={<left> <lower> <right> <upper>}
	%\includegraphics[trim={4cm 8cm 4cm 8cm},clip,width=.49\textwidth]{Images/LQG_weighted4.pdf}
	\includegraphics[trim={4cm 8cm 4cm 8cm},clip,width=1\textwidth]{Images/BibGood.pdf}
	\caption{Picture of a library, $\delta = 4.32$}
	\label{sub:BibGood}
\end{subfigure}
	\begin{subfigure}{0.49\textwidth}
	\centering
	%trim={<left> <lower> <right> <upper>}
	%\includegraphics[trim={4cm 8cm 4cm 8cm},clip,width=.49\textwidth]{Images/LQG_weighted4.pdf}
	\includegraphics[trim={4cm 8cm 4cm 8cm},clip,width=1\textwidth]{Images/BibMid.pdf}
	\caption{Picture of a library, $\delta = 43.205$}
	\label{sub:BibMid}
\end{subfigure}
	\begin{subfigure}{0.49\textwidth}
	\centering
	%trim={<left> <lower> <right> <upper>}
	%\includegraphics[trim={4cm 8cm 4cm 8cm},clip,width=.49\textwidth]{Images/LQG_weighted4.pdf}
	\includegraphics[trim={4cm 8cm 4cm 8cm},clip,width=1\textwidth]{Images/BibBad.pdf}
	\caption{Picture of a library, $\delta = 432.048$}
	\label{sub:BibBad}
\end{subfigure}
	\caption{Picture of a library (taken from personal phone), for different values of hard threshold. Using BiorSplines4.4 as wavelet and decomposition level of 4}
	\label{fig:Bib}
\end{figure}

    \subsubsection{Question 2.5}

	We will now add zero-mean, Gaussian white noise with variance of 0.01 to \cref{sub:Bib} via matlab \texttt{imnoise(I,'gaussian')} function. The noisy image looks like \cref{fig:Noisy}, it has an SNR (signal to noise ratio) of 7.41 and clearly looks noisy. The SNR was computed via
	\begin{equation}
		\texttt{SNR} = 10 \log_{10}{\frac{\|\hat{A}\|_F}{\|\hat{A}-A\|_F}}
	\end{equation}
	With $A$ the tensor representing the image, $\hat{A}$ the reconstructed image, in the above case $\hat{A}$ is the noisy image.
    \begin{figure}[H]
	\centering
	\includegraphics[trim={0cm 6cm 0cm 5.5cm},clip,width=0.7\textwidth]{Images/NoisyBib.pdf}
	\caption{\cref{sub:Bib} with added zero-mean, Gaussian white noise with variance of 0.01. Added via matlab \texttt{imnoise(I,'gaussian')}}
	\label{fig:Noisy}
\end{figure}

	We will test the denosing procedure for both hard and soft thresholding, for a range of $\delta = p T$'s, with $T$ the max of the wavelet coefficients of the 3 RGB colors and $p \in [0,1]$. 100 equidistant $p$'s were tested, obtained from matlab \texttt{linspace(0,1,100)}. \\
	The tested wavelets range from (For 4 level deep)
	\begin{itemize}
		\item Daubechies (orthogonal): \texttt{"db1",...,"db45"}
		\item BiorSplines (biorthgonal): \texttt{"bior1.1", "bior1.3", "bior1.5"
			"bior2.2", "bior2.4", "bior2.6", "bior2.8"
			"bior3.1", "bior3.3", "bior3.5", "bior3.7"
			"bior3.9", "bior4.4", "bior5.5", "bior6.8"}	
		\item ReverseBior (biorthogonal): \texttt{	"rbio1.1", "rbio1.3", "rbio1.5"
			"rbio2.2", "rbio2.4", "rbio2.6", "rbio2.8"	
			"rbio3.1", "rbio3.3", "rbio3.5", "rbio3.7"
			"rbio3.9", "rbio4.4", "rbio5.5", "rbio6.8"}
	\end{itemize}
	1500 different combinations were tested, only the best will be shown (for obvious reasons) \footnote{To run the tests and recreate \cref{fig:Bib,fig:Noisy,fig:RecBib,fig:CusBib}, run \texttt{ex2point2.m}} \\
	
	The best parameters will be determined using the SNR metric (the higher the better), of course in practice the clean image is usually not available, so the "eyeball" norm would have to be use. The best parameters according to \texttt{SNR} are: \texttt{rbio2.4} for the wavelet, threshold of 3.89 and a $\texttt{SNR} = \texttt{15.36}$, using hard thresholding. See \cref{fig:RecBib} for the resulting recontructed image. It looks a little more whitened, but otherwise looks the same as the noisy figure. The results are pretty underwhelming.

    \begin{figure}[H]
	\centering
	\includegraphics[trim={0cm 6cm 0cm 5.5cm},clip,width=0.7\textwidth]{Images/RecontructedBib.pdf}
	\caption{Recontructed image from \cref{fig:Noisy}. Using threshold of 3.89, hard thresholding and \texttt{rbio2.4} for the wavelet}
	\label{fig:RecBib}
\end{figure}
	
	By trial and error, we get the following parameters: threshold of 41.71, \texttt{db30} as wavelet and \texttt{SNR = 6.7} using soft thresholding. As we can see in \cref{fig:CusBib}, the image feels less noisy but more blurry.
	
    \begin{figure}[H]
	\centering
	\includegraphics[trim={0cm 6cm 0cm 5.5cm},clip,width=0.7\textwidth]{Images/CustomBib.pdf}
	\caption{Recontructed image from \cref{fig:Noisy}. Using threshold of 41.71, soft thresholding and \texttt{db30} for the wavelet}
	\label{fig:CusBib}
	
	The better image is a matter of preference.
	
\end{figure}
	
	\begin{comment}
\begin{table}[H]
	\centering
	\begin{tabular}{|l|l|l|l|l|l|l|l|l|l|l|}
		\hline
		$p$	& 0.001 &   0.11 &   0.22  &  0.33 &   0.44 &   0.56  &  0.68 &   0.78  &  0.89   & 1 \\ \hline
		\texttt{db1} & & & & & & & & & & \\ \hline
	\end{tabular}
	\caption{Table summarizing the SNR results for different threshold values, tested on \cref{sub:Bib}. Using BiorSplines4.4 as wavelet and decomposition level of 4}
	\label{tab:ImageNoise}
\end{table}
	\end{comment}
	
	
	
	%However the parameters are not generalizable since these are the best parameters for this denoising this specific image with gaussian noise, for different type of noise and image the optimal parameters might be different.
	
    \subsection{Using a redundant wavelet transform}

    \subsubsection{Task 2.6}

    \subsubsection{Question 2.7}

    \subsubsection{Question 2.8}

    \subsubsection{Task 2.9}

    \section{Wavelet-based inpainting}

    \subsection{An iterative algorithm}

    \subsubsection{Task 3.1}

    \subsubsection{Question 3.2}

    \subsubsection{Question 3.3}

    \subsubsection{Question 3.4}

 \end{document}