\documentclass[a4paper]{article}
\usepackage{float}
\usepackage[ruled,vlined,linesnumbered,algo2e]{algorithm2e}
\usepackage{amsmath,amssymb}
\usepackage{makecell}
\usepackage{tikz}
\usepackage[margin=0.8in]{geometry}
%\usepackage{biblatex} %Imports biblatex package
%\addbibresource{NLA.bib}
\DeclareMathOperator*{\argmax}{arg\,max}
\DeclareMathOperator*{\argmin}{arg\,min}
\DeclareMathOperator*{\diag}{diag}
\DeclareMathOperator*{\trace}{trace}
\DeclareMathOperator*{\sign}{sign}

\newcommand*\circled[1]{\tikz[baseline=(char.base)]{
		\node[shape=circle,draw=red,inner sep=1pt] (char) {#1};}}
\newcommand{\mbf}[1]{\mathbf{#1}}
\setlength\parindent{0pt} %% Do not touch this
\usepackage{amsthm}
\newtheorem{lemma}{Lemma}
\newtheorem{theorem}{Theorem}
%% -----------------------------
%% TITLE
%% -----------------------------
\title{Homework: Regularization} %% Assignment Title
%% Change "\today" by another date manually
%% -----------------------------
%% -----------------------------

%% %%%%%%%%%%%%%%%%%%%%%%%%%
\newcommand\myemptypage{
	\null
	\thispagestyle{empty}
	\addtocounter{page}{-1}
	\newpage
}

\usepackage{fancyhdr}
\usepackage{listings}
%\usepackage{siunitx}
\usepackage{hyperref}
\usepackage{etoolbox,refcount}
\usepackage{multicol}
\usepackage{tabularx,colortbl}
\usepackage{lastpage}
\usepackage{pgfplots}
\pgfplotsset{compat = newest}
\usepackage{biblatex} %Imports biblatex package
%\addbibresource{NLA.bib}
\usepackage{diagbox}
\pagestyle{fancy}
\usepackage{caption}
\usepackage{subcaption}
\usepackage{mathtools}
\usepackage{cleveref}
\usepackage[bottom]{footmisc}
\usepackage{comment}

\fancyhead[C]{}
\fancyhead[r]{}
%\renewcommand{\footrulewidth}{1pt}
\fancyfoot[L]{}
\fancyfoot[C]{}
\fancyfoot[R]{\thepage\ / \pageref*{LastPage}}

\begin{document}
	
	\begin{titlepage}
		
		\newcommand{\HRule}{\rule{\linewidth}{0.5mm}}  
		
		\begin{center}
			
			%\includegraphics[scale=0.3]{Documents/KUL_ENG.png}\\[1cm]
			\textsc{\LARGE Faculty of Engineering Science}\\[1.5cm] 
			
			
			\HRule \\[0.1cm]
			
			{\huge{ \bfseries Denoising and inpainting with wavelets}} 
			\vspace{5pt}
			{\\ \Large{\bfseries Wavelets with application in signal and Image processing}}
			\HRule \\[1cm]
			
			%\includegraphics[width=0.3\textwidth]{images1/INEOS.png}\\
			\vspace*{0.1cm}
			
			\begin{minipage}{0.4\textwidth}
				\textit{Auteur}
				\begin{flushleft} \large
					\begin{enumerate}
						\item[] \textsc{Thakur} Arnav\\
                        \item[] \textsc{Mohamed Yassin} Arman\\

						
					\end{enumerate} 
				\end{flushleft}
			\end{minipage}\\[1cm]
			\begin{minipage}{0.4\textwidth}
				\textit{Professors}
				\begin{flushleft}
					\begin{enumerate}
						\item[] Prof. Daan Huybrechs\\
					\end{enumerate} 
				\end{flushleft}
			\end{minipage}\\
			
			\vspace*{1.5cm}
			
			\includegraphics[width=0.5\textwidth]{Images/KUL_Eng_logo.png}\\
			{\Large{Academic year 2023 - 2024}}
			
		\end{center}
	\end{titlepage}
	
	\newpage
	
	\tableofcontents
	
	\newpage

    \section{Wavelet-based denoising}
 
	\subsection{A univariate functions with noise}

    \subsubsection{Question 2.1}

	Consider the function:
    \begin{equation*}
	f(x) = (2+\cos{(x)}) |x| \sign{(x-1)}
	\end{equation*}
    Function sampled in a set of $N=1000$ points in the interval $[-2,2]$
    \begin{align*}
    	f_j = f(x_j) && x_j = -2 + 4\frac{j-1}{N-1}, \quad j=1,\ldots,N
    \end{align*}
    
    We compute the wavelet transform of the vector $f = \{f_j\}_{j=1}^N$ using the Daubechies 2 wavelet of 4 levels deep. boundary conditions used is: Symmetrization (half-point). 
    \begin{figure}[H]
	\centering
	\includegraphics[trim={3.5cm 8cm 4cm 9cm},clip,width=0.7\textwidth]{Images/Coefficents.pdf}
	\caption{Coefficients of Wavelet transform (4 levels deep, using the Daubechies 2, $N=1000$ sample points, between $[-2,2]$. DWT Extension Mode: Symmetrization (half-point)}
	\label{fig:Coeff}
    \end{figure}

    We can see in \cref{fig:Coeff}, that the size of the coefficients decreases as $i$ increases. The meaning of this is that the coefficients of small size are of less importance to the reconstruction of the signal, and thus removing them will disproportionally remove noise.

    \subsubsection{Task 2.2}

	Noise is added to the signal $\tilde{f_i} = f_i + \epsilon \mathcal{U}(0,1)$ ($\hat{f}$ is the reconstructed $\tilde{f}$). We chose $\epsilon = \texttt{1e-1}$. \\
	Both hard thresholding:
	\begin{equation*}
			t_\delta = \begin{cases}
				0, \quad |x| < \delta\\
				x, \quad |x| \ge 0
		\end{cases}
	\end{equation*}
	And soft thresholding
	\begin{equation*}
			t_\delta = \begin{cases}
			0, \quad |x| < \delta\\
			\sign{(x)} (|x|-\delta), \quad |x| \ge 0
	\end{cases}
	\end{equation*}
	Are going to be tested. \\
	First let us choose $\delta = 0$, the results are in \cref{fig:Delta=0}. We can see in \cref{sub:ErrCoeffD=0}, that the highest error is in the first few coefficients. Which makes sense since we are in some sense moving the mean of the function $f$ by adding a value sampled from the uniform distribution, $\rightarrow$ lower frequencies should be more effected. The total error $E = \|f-\hat{f}\|_2 = 1.805$.\\
	
	Now let us set $\delta = \texttt{1e-1}$, the resulting figures are \cref{fig:Delta=0.1}. 3.35\% of the wavelet coefficients were set to 0, for both types of thresholding. \\
	$E = 1.5643$, which is indeed lower then the previous one. This seems manly due to the fact that $\hat{f}$ is smoother then $\tilde{f}$, see \cref{sub:ErrorD=0.1}. \\
	If knowledge of the mean of the noise is available, we can come put a better reconstruction of $f$ by $\hat{f}-mean(noise)$, in fact this is what will be done for the rest of the subsection, the resulting $E = 0.1982$, see \cref{sub:Delta=0.1Better}. In \cref{sub:Delta=0.1BetterSoft}, we can see the error when using soft thresholding, $E = 0.61$.

\begin{figure}[H]
	\centering
	\begin{subfigure}{0.49\textwidth}
	\centering
	%trim={<left> <lower> <right> <upper>}
	%\includegraphics[trim={4cm 8cm 4cm 8cm},clip,width=.49\textwidth]{Images/LQG_weighted4.pdf}
	\includegraphics[trim={4cm 8cm 4cm 8cm},clip,width=1\textwidth]{Images/FuncNoise.pdf}
	\caption{Plot of function, and it's noisy counterpart}
	\label{sub:FuncNoiseD=0}
\end{subfigure}
	\begin{subfigure}{0.49\textwidth}
		\centering
		%trim={<left> <lower> <right> <upper>}
		%\includegraphics[trim={4cm 8cm 4cm 8cm},clip,width=.49\textwidth]{Images/LQG_weighted4.pdf}
		\includegraphics[trim={4cm 8cm 4cm 8cm},clip,width=1\textwidth]{Images/CoeffDelta=0.pdf}
		\caption{The error between the real coefficients and noisy ones}
		\label{sub:ErrCoeffD=0}
	\end{subfigure}
	\begin{subfigure}{0.49\textwidth}
		\centering
		\includegraphics[trim={4cm 8cm 4cm 8cm},clip,width=1\textwidth]{Images/Delta=0.pdf}
		\caption{The error between real signal and noisy one}
		\label{sub:ErrorD=0}
	\end{subfigure}
	\caption{Plots for $\delta = 0$, black line is mean of noise. For hard thresholding}
	\label{fig:Delta=0}
\end{figure}

\begin{figure}[H]
	\centering
	\begin{subfigure}{0.49\textwidth}
		\centering
		%trim={<left> <lower> <right> <upper>}
		%\includegraphics[trim={4cm 8cm 4cm 8cm},clip,width=.49\textwidth]{Images/LQG_weighted4.pdf}
		\includegraphics[trim={4cm 8cm 4cm 8cm},clip,width=1\textwidth]{Images/CoeffDelta=0.1.pdf}
		\caption{The error between the real coefficients and reconstructed ones}
		\label{sub:ErrCoeffD=0.1}
	\end{subfigure}
	\begin{subfigure}{0.49\textwidth}
		\centering
		\includegraphics[trim={3.5cm 8cm 4cm 8cm},clip,width=1\textwidth]{Images/Delta=0.1.pdf}
		\caption{The error between real signal and reconstructed one}
		\label{sub:ErrorD=0.1}
	\end{subfigure}
	\caption{Plots for $\delta = 0.1$, black line is mean of noise. For hard thresholding}
	\label{fig:Delta=0.1}
\end{figure}

    \begin{figure}[H]
	\begin{subfigure}{0.49\textwidth}
	\centering
	\centering
\includegraphics[trim={3.5cm 8cm 4cm 9cm},clip,width=1\textwidth]{Images/Delta=0.1Better.pdf}
\caption{The error between real signal and reconstructed one, black line is mean of noise. For hard thresholding}
\label{sub:Delta=0.1Better}
\end{subfigure}
\begin{subfigure}{0.49\textwidth}
	\centering
\includegraphics[trim={3.5cm 8cm 4cm 9cm},clip,width=1\textwidth]{Images/Delta=0.1Soft.pdf}
\caption{The error between real signal and reconstructed one, black line is mean of noise. For soft thresholding}
\label{sub:Delta=0.1BetterSoft}
\end{subfigure}
\caption{}
\end{figure}


    \subsubsection{Question 2.3}

	Now let us try to find the best parameter $\delta$ in order to minimize the noise. This will be done by simply checking MSE between the real and filtered coefficients $\sum_{i} (x_i - \hat{x}_i)^2$, for 101 different $\delta$'s in $[\texttt{1e-10},\texttt{1e0}]$. The results are shown in \cref{fig:OptiDelta}, the best delta according to MSE is 
	\begin{itemize}
		\item Hard threshold: $\delta = 0.501$ with an MSE=0.0093
		\item Soft threshold: $\delta = 0.0398$ with an MSE=0.0097
	\end{itemize}
	The reconstructed functions do have less error then the noisy one. \\
	For hard thresholding, the error is especially noticeable for $x = 0$ (\cref{sub:BestDelta,sub:BestDeltaFunc}), probably due to the fact that $f$ is not differentiable due to $|x|$, and because of the very discontinuous nature of the thresholding this results in more error. For $x = 1$, we can also see a slight bump in error due the discontinuation of the $\sign$ function.\\
	For soft thresholding it would seem that the error is higher over the whole domain but lower in $x=0$, due to the fact that it is less discontinuous then hard thresholding. (\cref{sub:BestDeltaSoft,sub:BestDeltaSoftFunc}) \\
	
	It would seem that hard thresholding is overall smoother then soft except for $x=0$. 

    \begin{figure}[H]
	\begin{subfigure}{0.49\textwidth}
	\centering
\includegraphics[trim={3.5cm 8cm 4cm 9cm},clip,width=1\textwidth]{Images/DeltaOpti.pdf}
\caption{The error between real signal and reconstructed one, black line is mean of noise. $\delta = 0.501$.For Hard thresholding}
\label{sub:BestDelta}
\end{subfigure}
\begin{subfigure}{0.49\textwidth}
	\centering
\includegraphics[trim={3.5cm 8cm 4cm 9cm},clip,width=1\textwidth]{Images/DeltaOptiSoft.pdf}
\caption{The error between real signal and reconstructed one, black line is mean of noise.$\delta = 0.0398$. For soft threasholding}
\label{sub:BestDeltaSoft}
\end{subfigure}
	\begin{subfigure}{0.49\textwidth}
	\centering
	\includegraphics[trim={3.5cm 8cm 4cm 9cm},clip,width=1\textwidth]{Images/HardFunc.pdf}
	\caption{Plot of clean and reconstructed signal. $\delta = 0.501$. For Hard thresholding}
	\label{sub:BestDeltaFunc}
\end{subfigure}
\begin{subfigure}{0.49\textwidth}
	\centering
	\includegraphics[trim={3.5cm 8cm 4cm 9cm},clip,width=1\textwidth]{Images/SoftFunc.pdf}
	\caption{Plot of clean and reconstructed signal. $\delta = 0.0398$. For soft threasholding}
	\label{sub:BestDeltaSoftFunc}
\end{subfigure}
\caption{Plots for the recontructed functions, via different type of thresholding}
\label{fig:OptiDelta}
\end{figure}

    \subsection{Images with noise}

    \subsubsection{Task 2.4}

	We will test the wavelet-based denoising scheme on a picture (\cref{sub:Bib}) for different threshold (using hard thesholding), the resulting figures are in \cref{fig:Bib}, \cref{tab:bib} summarizes the results. We can clearly see as the threshold is increased, the MSE and compression ration increase as well. This is expected as a higher threshold results in more of the (signal) picture being thrown away.

\begin{table}[H]
	\centering
	\begin{tabular}{|l|l|l|l|}
	\hline
	$\delta$	& Compression ratio & MSE & relative MSE \\ \hline
	4.32	& 1.95 & 0.89 & \texttt{3.62e-6} \\ \hline
	43.205	& 9.73 & 6.43 & \texttt{2.63e-5} \\ \hline
	432.048	& 218.84 & 8.53 & \texttt{3.49e-5} \\ \hline
	\end{tabular}
	\caption{Table summarizing the results for different threshold values, tested on \cref{sub:Bib}}
	\label{tab:bib}
\end{table}

\begin{figure}[H]
	\centering
	\begin{subfigure}{0.49\textwidth}
		\centering
		%trim={<left> <lower> <right> <upper>}
		%\includegraphics[trim={4cm 8cm 4cm 8cm},clip,width=.49\textwidth]{Images/LQG_weighted4.pdf}
		\includegraphics[trim={4cm 8cm 4cm 8cm},clip,width=1\textwidth]{Images/Bib.pdf}
		\caption{Picture of a library, original}
		\label{sub:Bib}
	\end{subfigure}
	\begin{subfigure}{0.49\textwidth}
	\centering
	%trim={<left> <lower> <right> <upper>}
	%\includegraphics[trim={4cm 8cm 4cm 8cm},clip,width=.49\textwidth]{Images/LQG_weighted4.pdf}
	\includegraphics[trim={4cm 8cm 4cm 8cm},clip,width=1\textwidth]{Images/BibGood.pdf}
	\caption{Picture of a library, $\delta = 4.32$}
	\label{sub:BibGood}
\end{subfigure}
	\begin{subfigure}{0.49\textwidth}
	\centering
	%trim={<left> <lower> <right> <upper>}
	%\includegraphics[trim={4cm 8cm 4cm 8cm},clip,width=.49\textwidth]{Images/LQG_weighted4.pdf}
	\includegraphics[trim={4cm 8cm 4cm 8cm},clip,width=1\textwidth]{Images/BibMid.pdf}
	\caption{Picture of a library, $\delta = 43.205$}
	\label{sub:BibMid}
\end{subfigure}
	\begin{subfigure}{0.49\textwidth}
	\centering
	%trim={<left> <lower> <right> <upper>}
	%\includegraphics[trim={4cm 8cm 4cm 8cm},clip,width=.49\textwidth]{Images/LQG_weighted4.pdf}
	\includegraphics[trim={4cm 8cm 4cm 8cm},clip,width=1\textwidth]{Images/BibBad.pdf}
	\caption{Picture of a library, $\delta = 432.048$}
	\label{sub:BibBad}
\end{subfigure}
	\caption{Picture of a library (taken from personal phone), for different values of hard threshold}
	\label{fig:Bib}
\end{figure}

    \subsubsection{Question 2.5}

    \subsection{Using a redundant wavelet transform}

    \subsubsection{Task 2.6}

    \subsubsection{Question 2.7}

    \subsubsection{Question 2.8}

    \subsubsection{Task 2.9}

    \section{Wavelet-based inpainting}

    \subsection{An iterative algorithm}

    \subsubsection{Task 3.1}

    \subsubsection{Question 3.2}

    \subsubsection{Question 3.3}

    \subsubsection{Question 3.4}

 \end{document}